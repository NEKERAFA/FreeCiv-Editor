Desde un tiempo a esta parte, hemos visto como la industria de los videojuegos ha experimentado un cambio drástico, haciendo que actualmente sea uno de los principales motores económicos a nivel mundial, teniendo casos en los que un proyecto de este campo supera en nivel económico y de producción a muchas obras de la industria del cine. Esto se debe a todos los avances tecnológicos que nutren al mundo de los videojuegos, así como la madurez de un medio en el que muchos autores han visto su reconocimiento no por la diversión que plantean sus obras, si no por llevar al videojuego a su máxima expresión, realizándolo desde una forma más creativa o desarrollando el contenido y el alma del mismo de un modo que pueda llegar a un gran público. \\

Con respecto al tema que nos trata, uno de los puntos que hacen crecer esta industria es la Inteligencia Artificial, ya que el poder que aporta esta rama de la informática para crear sistemas eficientes o que se comporten de una manera inteligente permite llevar al videojuego a un nuevo nivel. Estos sistemas permiten crear desde interacciones humano-máquina más naturales al aplicarlas a las distintas entidades que pueden conformar el universo de un videojuego, como enemigos inteligentes que aprenden la forma de jugar del usuario o entidades aliadas que ajustan su comportamiento a la experiencia del jugador; como aplicar una variedad y diversidad al ecosistema al usar elementos de programación evolutiva a entidades \cite{5286468} o generación procedimental \cite{parkin_2016} para generar escenarios. Incluso hemos visto este año como el aprendizaje profundo o \textit{deep learning} ha permitido que empresas de \textit{hardware} como Nvidia han construido su nueva generación de tarjetas gráficas para el mundo de los videojuegos con la premisa de un avance muy significativo en la calidad del renderizado de escenas generadas por computador, ya que permiten realizar una técnica muy costosa como puede ser el trazado de rayos o \textit{raytracing} \cite{Whitted:1980:IIM:358876.358882} en tiempo real \cite{Parker:2013:GRT:2447976.2447997}. \\

Volviendo a la generación procedimental de entornos, muchos de estos sistemas se basan en una generación pseudo-aleatoria de puntos en donde se crean distintos patrones ya definidos por un ser humano. Esto se repite, incluso aplicando transformaciones de estos patrones hasta generar un mapa que parezca real en un alto grado. El problema de estos sistemas llega al momento de crear un entorno grande, resultando extremadamente lentos, por lo que otra aproximación muy usada es usar funciones matemáticas que definen el contorno del terreno, pudiendo incluso generarlo infinitamente en tiempo real a medida que el usuario avanza. Estas aproximaciones realizan un gran trabajo en cuanto a eficiencia y rapidez, más resultan ineficientes a la hora de plantear requisitos y modificaciones concretas, ya que muchos de los elementos que definen estos elementos están expuestos a la incertidumbre debido a su propia construcción. Este proyecto se centra en la aplicación de otro tipo de aproximación para la resolución de este problema, empezando por un caso concreto de generación de entornos para un sistema de entretenimiento.

\section{Motivación}

Como ya se ha comentado anteriormente, muchos de los campos de la Inteligencia Artificial se están aplicando cada vez más en la industria de los videojuegos, en especial para facilitar la tarea de crear sistemas y entornos que sean orgánicos y naturales para el consumidor de este tipo de \textit{software}. Para ello se han aplicado muchas aproximación y optimizaciones de cara a tener sistemas lo más flexibles y con el objetivo de que respondan en un tiempo razonable consumiendo los mínimos recursos posibles. Esto último viene a requisito de que un videojuego tiene que ser interactivo y con una tasa de respuesta lo más pequeña posible, así como poder ser ejecutado en sistemas con recursos muy escasos, como puede ser el caso de una consola portátil o un teléfono inteligente. \\

A pesar de esto, muchos de los sistemas presentan problemas en el momento de ser modificados o de razonar una serie de soluciones. Es por esto que muchas veces, una modificación pequeña de los parámetros internos puede llegar a ocasionar que el resultado varíe enormemente, haciendo incluso que en ciertos videojuegos que requieren de una conexión a Internet para ser ejecutados, tengan que reiniciar el escenario, haciendo que sus usuarios pierdan el progreso explorado en el mapa. Ejemplos de esto los tenemos en muchos servidores del videojuego \textit{Minecraft} o en las actualizaciones del videojuego \textit{No Man's Sky}. Debido a esto, muchas empresas prescinden de realizar actualizaciones a las partes del código que controlan la generación del universo, lastrando consigo problemas que pueden ocasionar fallos por la mala gestión que pudo ocurrir a la hora de diseñar el sistema. \\

En este proyecto se seguirá una aproximación distinta, ya que parte de un modelo declarativo que consiste en definir la construcción del entorno en vez del mismo. Para esta tarea se usará una herramienta matemática concreta, la lógica proposicional, con la que se podrá definir un conjunto de reglas y restricciones con el que el sistema podrá completar y rellenar partes del entorno para generar un mapa final. Con la idea de restringir más este dominio, el sistema se centrará en la generación de un entorno en una rejilla 2D para el videojuego Freeciv, el cual al ser de formato libre, se puede encontrar fácilmente información de como funciona un mapa conforme a los requisitos del juego. \\

La gran ventaja que plantea el uso de estas técnicas es la independencia de la definición del modelo lógico con respecto al algoritmo de búsqueda o heurística que se use para hallar las posibles soluciones finales. Partiendo de esto, se usará el paradigma de \textit{Answer Set Programming}, una variante de Programación Lógica que se ha convertido hoy en día en uno de los lenguajes de representación de conocimiento con mayor proyección y difusión debido tanto a su eficiencia en aplicación práctica para la resolución de problemas como a su flexibilidad y expresividad para la representación del conocimiento. La generación de escenarios de videojuegos supone un desafío como caso de prueba para \textit{Answer Set Programming}, ya que el número de combinaciones posibles aumenta exponencialmente en función del tamaño del escenario. Cabe remarcar que ya existe antecedendes de generación declarativa para el diseño de espacios o entornos \cite{desing} usando este paradigma, así como el uso de \textit{Answer Set Programming} en otros ámbitos donde también ocurre el mismo problema combinatorio, como puede ser la composición musical \cite{haspie} \cite{DBLP:journals/corr/abs-1006-4948}.

\section{Objetivos}

Teniendo en cuenta lo explicado anteriormente, los objetivos finales de este proyecto son los siguientes:

\begin{itemize}
	\item \textbf{Definición de un modelo declarativa para generación de escenarios:} Este proyecto se propone definir un modelo declarativo que de respuesta a como construir un escenario del videojuego Freeciv usando las técnicas y paradigmas de \textit{Answer Set Programming}, definiendo un conjunto de reglas, expresadas como restricciones en programación lógica, de modo que el usuario pueda variar sustancialmente la configuración de los escenarios obtenidos en función de la representación que se haga del problema.
	\item \textbf{Construcción de una pequeña herramienta gráfica:} Este pequeño editor gráfico permitirá al usuario crear y manipular manualmente el escenario, pudiendo fijar también algunas partes de la configuración del escenario y dejando que la herramienta complete las zonas no definidas.
	\item \textbf{Eficiencia:} Debido al problema que supone este tipo de aproximación, sobre todo al definir mapas de gran tamaño, la herramienta debe tener en cuenta la explosión combinatoria que puede ocasionar a la hora de obtener una solución. Así mismo, la herramienta ha de dar la respuesta en un tiempo tal que el usuario sienta razonable.
\end{itemize}

\section{Estructura de la memoria}

Para tener en cuenta la estructura que seguirá la memoria del presente proyecto, a continuación se explica los capítulos en los que se divide esta memoria:

\begin{itemize}
	\item \textbf{\hyperlink{introduccion}{Capítulo \ref*{introduction}. Introducción:}} Sirve como punto inicial a la lectura y conocimiento de este proyecto, describiendo la motivación que lo impulsa y detallando los objetivos a alcanzar.
	\item \textbf{\hyperlink{background}{Capítulo \ref*{background}. Contexto:}} Enmarca conceptos que son necesarios para entender el proyecto desarrollado, definiendo el plano tecnológico actual. Así mismo se describe y justifica las principales tecnologías empleadas en el desarrollo del mismo.
	\item \textbf{\hyperlink{mainwork}{Capítulo \ref*{mainwork}. Trabajo desarrollado:}} Explica las técnicas y el proceso de ingeniería llevado a cabo para la gestión, desarrollo y construcción del sistema propuesto para este proyecto. En este capítulo se indica el funcionamiento interno de la utilidad declarativa creada a la hora de plantear este proyecto.
	\item \textbf{\hyperlink{evaluation}{Capítulo \ref*{evaluation}. Evaluación:}} Realiza un análisis de las pruebas de rendimiento llevadas a cabo una vez acaba la construcción del proyecto, indicando la metodología y explicando los resultado obtenidos.
	\item \textbf{\hyperlink{conclusions}{Capítulo \ref*{conclusions}. Conclusiones:}} Ofrece una visión global de la viabilidad y calidad del sistema obtenido, así como se indican las vías de trabajo futura que se abre a la finalización del mismo.
	\item \textbf{\hyperlink{appendices}{Apéndices}:} Adjunta las siguientes secciones complementarias:
	\begin{itemize}
		\item \textbf{\hyperlink{bibliography}{Apéndice A. Bibliografía:}} Recoge la documentación bibliográfica sobre la que se apoya este proyecto.
	\end{itemize}
\end{itemize}

\section{Plan de trabajo}

Para el desarrollo del proyecto se han seguido las siguientes etapas:

\begin{itemize}
	\item Estudio y análisis del estado del arte sobre la generación de escenarios en entornos similares, así como la investigación y estudio de la documentación del software elegido para el proyecto. 
	\item Diseño e implementación del modelo declarativo para la generación de entornos en Freeciv.
	\item Diseño y construcción de una herramienta para la manipulación de entornos en FreeCiv.
	\item Diseño e implementación de un módulo que traduzca el resultado de ASP a Freeciv.
	\item Construcción de un componente capaz de evaluar el modelo lógico con distintos resultados.
	\item Evaluación de eficiencia para distintos casos de prueba.
	\item Redacción de la memoria del proyecto final. Esta fase se ha intentado realizar de forma paralela al resto de etapas.
\end{itemize}