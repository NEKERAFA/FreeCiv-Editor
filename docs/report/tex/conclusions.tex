En este proyecto se construido una herramienta declarativa con la que elaborar, mediante programación lógica, nuevos escenarios para el videojuego Freeciv. Para ello se usó el paradigma de \textit{Answer Set Programming}, una variante de Programación Lógica de uso frecuente para la Representación del Conocimiento y la resolución de problemas. La principal ventaja de \textit{Answer Set Programming} para este caso fue la facilidad que otorgaba el uso de predicados simples a la hora de definir ciertas propiedades dentro de un escenario concreto, así como añadir directamente reglas de restricción para ciertos elementos del mapa bajo la forma de reglas de programación lógica. Esto proporcionó una gran flexibilidad, ya que con ello sólo se necesita realizar la especificación del problema sin importar el método de resolución que se aplicará. \\

También se ha reducido el problema que surge al tener en cuenta todas las reglas y restricciones del mapa, lo cual produce una explosión combinatoria de soluciones, y ocasionaba que el sistema tardase en encontrar. Para ello se ha separado el problemas de búsqueda del modelo declarativo en distintos módulos independientes: un generador de regiones, un módulo para rellenar las regiones, un generador de biomas, un módulo que define cordilleras, un módulo para definir las celdas de agua y por último un módulo para definir los puntos iniciales de los jugadores. \\

A pesar de todo esto, existen algunas limitaciones de cara la finalización de este proyecto, las cuales pueden ser resueltas en un futuro inmediato:

\begin{itemize}
	\item El sistema no tiene en cuenta todos los elementos que proporciona el videojuego, por lo que no soporta actualmente la generación de ríos, ni la generación de recursos ni la generación de puntos de inicio para bárbaros o aldeas perdidas. Esto se podría solucionar añadiendo nuevos módulos a la generación del mismo.
	\item El sistema presenta graves problemas de eficiencia a la hora de generar grandes porciones de mapas, mermando la funcionalidad de la herramienta. Esto se puede observar en la Sección \ref{evaluation}, en donde había problemas con mapas mayores de 45x45 celdas en cuanto a tiempo de ejecución. Esto se podría resolver intentando pulir las reglas actuales de los módulos o replanteando la forma en la que se han definido, en algunos casos dividiendo más para reducir la carga de estos.
\end{itemize}

Para finalizar, se puede marcar una serie de líneas, tanto para solucionar y mejorar el sistema propuesto como para ampliar y extender las funcionalidades actuales de cara a largo plazo:

\begin{itemize}
	\item Mejorar la interfaz gráfica, incluyendo distintos elementos gráficos con los que manejar de forma más efectiva el mapa, como puede ser herramientas para ampliar y disminuir el mapa, mover el mapa o marcar de forma más visual el tipo de terreno marcar.
	\item Indicar de forma más visual restricciones en el mapa, así como zonas que tendrá en cuenta el generador para completar. Estas zonas incluso pueden tener restricciones de que piezas no poner o preferencias de los tipos de terrenos a generar.
	\item Publicar y anunciar la herramienta para tener una buena base de usuarios que puedan producir un \textit{feedback} del uso y características de la misma.
	\item Añadir a la base de conocimiento estilos personalizados, ya sea incluyéndolos dentro de las reglas o importándolos como perfiles. Estos permitirían que un usuario pueda tener preferencias de cara a la generación (que no le guste generar cordilleras cerca del agua, que los usuarios estén lo más cercano a bosques o recursos, etc).
\end{itemize}